\documentclass[conference]{IEEEtran}
\IEEEoverridecommandlockouts
% The preceding line is only needed to identify funding in the first footnote. If that is unneeded, please comment it out.
\usepackage{cite}
\usepackage{amsmath,amssymb,amsfonts}
\usepackage{algorithmic}
\usepackage{graphicx}
\usepackage{textcomp}
\usepackage{xcolor}
\def\BibTeX{{\rm B\kern-.05em{\sc i\kern-.025em b}\kern-.08em
    T\kern-.1667em\lower.7ex\hbox{E}\kern-.125emX}}
\begin{document}

\title{IEEE PAPER ON COMPANY WORKFLOW MAPPING SYSTEM\\
\thanks{COEP Tech University .}
}

\author{\IEEEauthorblockN{Sarvesh Mankar}
\IEEEauthorblockA{\textit{Computer Engineering, College of Engineering, Pune} \\
\textit{SPPU, Autonomous}\\
Pune, India \\
mankarsarvesh2543@gmail.com}
\and
\IEEEauthorblockN{Sarvesh Mankar}
\IEEEauthorblockA{\textit{Computer Engineering, College of Engineering, Pune} \\
\textit{SPPU, Autonomous}\\
Pune, India \\
mankarsarvesh2543@gmail.com}

 }

\maketitle

\begin{abstract}
Training is the most important activity, which plays an important role in the 
development of human resources. To put the right man at the right place with the trained personnel has become essential in today’s globalized market. No organisation has a choice on whether or not to develop employees. Human Resources are the lifeblood of any organization. 
Only through well-trained personnel, can an organization achieve its goals.
Human resource practices play a key role in attracting, motivating, rewarding and 
retaining employees.Business organizations realize more than ever that employee knowledge gained through training and development has become a strategic necessity and more the sources of strategic advantage. To provide system procedures for structured identification of training needs of personnel, for evaluation of effectiveness of training provided and for achieving competency.Training and Competency system aims to provide a platform for top management authorities to view and evaluate the working of various trainees and their progress in respective training programs. The major aim of Skill Space is to evaluate the competence level of a particular trainee and to provide the required knowledge or assistance. 

\end{abstract}

\begin{IEEEkeywords}
Human Resource, Competence, Training Programs, APQP ( Advanced Product Quality 
Planning) , Information Management, QMS (Quality Management System), Skill Analysis, Learning Management, Customer Relationship Management (CRM), Database 
Management
\end{IEEEkeywords}

\section{Introduction}
      
Training allows employees to acquire new skills, sharpen existing ones, perform better, 
increase productivity and be better leaders. Since a company is the sum of what employees 
achieve individually, organizations should do everything in their power to ensure that 
employees perform at their peak.
Competency management can benefit employees by allowing them more opportunities 
to increase their skills and aid in their professional development. These benefits also help the 
entire company grow and succeed. According to the Association for Talent Development 
(ATD), companies who invest in advanced training tools, like competency management 
software, enjoy a 218 percent higher income per employee compared to companies that do not.
The HR will convey the requirements or pre-requisites for the training. The training 
programs provided to trainees will be conducted by HR and prepared by DH and HR, which 
will be then displayed to trainees on to training schedules. The training needs can also be 
identified from time to time based on competence gaps identified, changing environmental 
requirements \& included in Quarterly schedules.

\section{Literature Review}

\subsection{Maintaining the Integrity of the Specifications}

According to JC Richards and T.S. Rogers (2014), CLIL and CBI are built around the core principles, one of which states that people learn a second language more succesfully when they use the language as a means of understanding content rather than language alone. It was cited by K. Elwood (2018) that CBI is an earlier model of CLIL And the term CBI and CLIL are interchangeably used in this paper. The findings of the study are presented in the subsequent section. Afterwards, the findings are discussed in light of relevant literature. Some concluding remarks are provided at the cod of the paper.

\section{Applications}

Training has become a necessity for all the organization in today’s dynamic business 
environment. It helps the employees to perform their jobs more efficiently and effectively 
resulting in long-term benefits for both the employees as well as the organizations. Training 
can be multi-dimensional. It can be specifically used to develop skills and knowledge that may 
be used at an Individual, Operational, Organizational level.


\section*{Conclusion and Future scope}

With new advances in artificial intelligence (AI), augmented reality (AR),
virtual reality (VR) and machine learning, technology is transforming the training industry at an accelerated pace. In the future, we could implement a skill matrix with the help of Artificial Intelligence and Machine Learning that would be useful to evaluate the knowledge and capabilities of an employee. Constant analysis would help the company reach heights.Recommendation Engines can be used to fascinate the employees to increase their skills by providing related training programmes based on their activity and interests. Hence, Skill Space will provide a platform for the trainees and trainers to place all training requirements and skill evaluation into one place, rather than working across multiple places.


	\begin{thebibliography} {}
	
	\bibitem {CODE}Company System, Sam-Tech ,https://sam-tech-2543.github.io/
	
	\bibitem{Wikipedia}JavaTpoint , https://www.javatpoint.com/nlp

	
	\end{thebibliography}
	

\end{document}
